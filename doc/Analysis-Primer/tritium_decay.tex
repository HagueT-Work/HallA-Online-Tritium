\subsection{Target composition}

Tritium decays to helium via the $\beta$-decay process $^3\text{H} \,\rightarrow\, ^3\text{He} + e^- + \overline{\nu}_e$, with a half-life of

\begin{equation*}
\tau_{1/2} = \ln(2)\tau = (4500 \pm 8) \text{ days} 
\end{equation*}

This results in a time-dependent target composition, with a decreasing (increasing) population of tritium (helium) nuclei.  These effects must be quantified and corrected in order to accurately extract the normalized tritium yield from tritium target data.

The target cell was filled with an initial tritium number density $n_T^0$, and initial helium number density $n_H^0$.  As tritium decays to helium, these number densities evolve in time as
\begin{align}
n_T &= n_T^0 \: e^{-t/\tau} \label{nT}\\[5pt]
n_H &= n_H^0(1 - e^{-t/\tau}) \label{nH},
\end{align}
where $t$ is the number of days since the target was filled.  Since the decay process preserves the total number of nuclei, the total number density $n_{tot}$ is constant in time:
\begin{align}
n_{tot} 	&= n_T + n_H \nonumber \\
		&= n_T^0 + n_H^0 
\end{align} 

With these quantities, the helium fraction can be defined:
\begin{equation}
f_H = \frac{n_H}{n_T + n_H} = \frac{n_H}{n_{tot}} \label{fH}
\end{equation}
Given an infinite amount of time, all of the tritium will decay to helium.  Therefore $f_H\rightarrow1$ as $t\rightarrow\infty$.  

\subsection{Normalized yield correction}

The normalized yield is defined as:
\begin{equation}
Y = \frac{N}{Qn},
\end{equation}
where $N$ is the number of detected electrons, $Q$ is the beam charge incident on the target, and $n$ is the target number density.  Assume that $N$ includes all corrections (deadtime, efficiency, endcap contamination, etc.) \textit{not} related to tritium decay.  In practice, the yield is extracted from multiple runs, so the number of detected electrons and luminosity must be summed over run number $i$:
\begin{equation}
Y = \frac{\sum N_i}{\sum Q_i n_i},
\end{equation}

The required correction must account not only for the evolution of the target composition (quantified in the previous section), but also for the fact that some of the detected electrons $N$ will have actually scattered from a helium nucleus instead of a tritium nucleus.  Begin by expressing the raw, uncorrected normalized yield (which is measured) as

\begin{equation}
Y_{raw} = \frac{\sum (T_i + H_i)}{\sum Q_i (n_{T,i} + n_{H,i})} \label{Yraw}
\end{equation}
where $T$ and $H$ are the number of detected electrons scattered by tritium and helium, respectively.  For time-dependent quantities (such as $n_{T,i}$ and $n_{H,i}$, given by Equations \ref{nT} and \ref{nH}), the subscript indicates the value of the quantity at the time of run $i$.  The goal is to obtain the normalized tritium yield $Y_T$ in terms of $Y_{raw}$ and correction factors, where

\begin{equation}
Y_T = \frac{\sum T_i}{\sum Q_i n_{T,i}}. 
\end{equation} 

Due to the helium contamination, the correction factor will depend on the normalized helium yield

\begin{equation}
Y_H = \frac{\sum H_i}{\sum Q_i n_{H,i}}. 
\end{equation}

From equation (\ref{Yraw}), only a few steps of algebra are required to obtain $Y_T$.  Recall that the total number density $n_{tot}=n_T + n_H$ is constant in time, and note that the tritium fraction $n_{T,i}/n = 1 - f_{H,i}$, where $f_H$ is the helium fraction defined by Equation \ref{fH}.

\begin{align*}
Y_{raw} 	&= \frac{\sum (T_i + H_i)}{\sum Q_i (n_{T,i} + n_{H,i})} \\[15pt]
		&= \frac{\sum T_i}{n_{tot} \sum Q_i} + \frac{\sum H_i}{n_{tot} \sum Q_i} \\[15pt]
		&= \left(\frac{\sum_i T_i}{\sum_i Q_i n_{T,i}}\right)\left(\frac{\sum_i Q_i n_{T,i}}{n_{tot} \sum_i Q_i}\right)
		+ \left(\frac{\sum_i H_i}{\sum_i Q_i n_{H,i}}\right)\left(\frac{\sum_i Q_i n_{H,i}}{n_{tot} \sum_i Q_i}\right) \\[15pt]
		&= Y_T\left(\frac{\sum Q_i(1-f_{H,i})}{\sum Q_i}\right) + Y_H\left(\frac{\sum Q_i f_{H,i}}{\sum Q_i}\right)
\end{align*}

To simplify notation, define the charge-averaged helium fraction:

\begin{equation}
\langle f_H \rangle \equiv \frac{\sum Q_i f_{H,i}}{\sum Q_i}
\end{equation}

Thus,

\begin{equation}
Y_{raw} = Y_T(1-\langle f_H \rangle) + Y_H \langle f_H \rangle,
\end{equation}

and finally,

\begin{equation}
Y_T = Y_{raw}\left(\frac{1}{1-\langle f_H \rangle}\right) - Y_H \left(\frac{\langle f_H \rangle}{1-\langle f_H \rangle}\right)
\end{equation}

\subsection{Uncertainty propagation}

Pending

